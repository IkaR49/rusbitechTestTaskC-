\documentclass [12pt, a4paper] {article}
\usepackage {amsmath, amsthm, amssymb}
\usepackage [utf8] {inputenc}
\usepackage [T1, T2A] {fontenc}
\usepackage [english, russian] {babel}

\title {Задание 3}
\author {Левченко Павел}

\begin{document}

\section {Постановка задачи}
Расписать необходимые матричные преобразования для перемещения, поворота и масштабирования треугольника в трёхмерной декартовой системе координат.

\section {Некоторые соглашения}
Сразу хочу договориться о некоторых соглашениях в математических рассуждениях.

Во-первых, для удобства записи и более простого воспринимания текста большинство промежуточных рассуждений можно производить в двумерном пространстве, так как особой разницы между двумерным и трёхмерным пространствами, при условии совпадения базисов, нет. Однако в каждом разделе конечные матрицы преобразования будут даны для трёхмерного пространства, что и требуется заданием.

Во-вторых, надеюсь от меня не требуют демонстрации умений ручного перемножения матриц, поэтому расписывать я их здесь не буду.

В-третьих, пусть нам дан треугольник с вершинами $A(a_x, a_y, a_z)$, $B(b_x, b_y, b_z)$, $C(c_x, c_y, c_z)$, который мы и будем в дальнейшем рассматривать.

\section {Масштабирование}
Операции масштабирования фигуры являются самыми простыми из всех рассматриваемых. Предположим, нам необходимо масштабировать наш треугольник на величину $k$ с сохранением пропорций.
Для этого достаточно матрицу, составленную из координат вершин исходного треугольника, умножить на скалярную величину $k$:
$\begin {pmatrix}
	a'_x & a'_y \\
	b'_x & b'_y \\
	c'_x & c'_y
\end {pmatrix} = k \begin {pmatrix}
	a_x & a_y \\
	b_x & b_y \\
	c_x & c_y
\end {pmatrix} = \begin {pmatrix}
	ka_x & ka_y \\
	kb_x & kb_y \\
	kc_x & kc_y
\end {pmatrix}$

Вполне очевидно, что умножение матрицы на скаляр равносильно умножению матрицы на единичную матрицу, помноженную на этот скаляр. Отсюда можно сделать вывод, что для непропорционального масштабирования возможно умножать исходную матрицу на матрицу, составленную из единичной, но вместо элементов главной диагонали - необходимые коэффициенты для абциссы, ординаты (и аппликаты): $\begin {pmatrix} k_x & 0 \\ 0 & k_y \end {pmatrix}$

Таким образом, конечное матричное уравнение для пространства $R^3$ примет вид:
\begin {equation} \label {eq:scale}
	\begin {pmatrix}
		a'_x & a'_y & a'_z \\
		b'_x & b'_y & b'_z \\
		c'_x & c'_y & c'_z
	\end {pmatrix} = \begin {pmatrix}
		a_x & a_y & a_z \\
		b_x & b_y & b_z \\
		c_x & c_y & c_z
	\end {pmatrix} \begin {pmatrix}
		k_x & 0 & 0 \\
		0 & k_y & 0 \\
		0 & 0 & k_z
	\end {pmatrix}
\end {equation}

\section {Перемещение}
Снова рассмотрим наш треугольник $ABC$, который с помощью вектора переноса $\overrightarrow{k}(k_x, k_y)^T$ должен стать треугольником с вершинами $A'(a'_x, a'_y)$, $B'(b'_x, b'_y)$, $C'(c'_x, c'_y)$.

Вполне очевидно, что для любой точки справедливо следующее:
$\left\{
	\begin {aligned}
		x' = x + k_x\\
		y' = y + k_y
	\end {aligned}
\right.$

Для того, чтобы подобные действия можно было совершать в матричном виде, доведём наши декартовы координаты до однородных, то есть для пространства $R^2$ добавим ещё одну строку и один столбец в матричную запись. Теперь исходный треугольник можно удобно записать в виде единой матрицы:
$\begin {pmatrix}
	A\\
	B\\
	C
\end {pmatrix} = \begin {pmatrix}
	a_x & a_y & 1\\
	b_x & b_y & 1\\
	c_x & c_y & 1
\end {pmatrix}$,
где последний столбец говорит о том, что каждая вершина треугольника является отображением точки начала координат в однородной системе координат.

Теперь координаты любой точки можно получить из матричного уравнения:
$\begin {pmatrix}
	x' & y' & 1
\end {pmatrix} = \begin {pmatrix}
	x & y & 1
\end {pmatrix} K$,
где $K = \begin {pmatrix}
	1 & 0 & 0\\
	0 & 1 & 0\\
	k_x & k_y & 1
\end {pmatrix}$ - матрица переноса (составленная из вектора $\overrightarrow{k}$).

Зная свойства перемножения матриц можно заметить, что умножение матрицы, строки которой составлены из координат вершин треугольника на матрицу переноса $K$, приведёт к достоверному результату для всех вершин, поэтому в конечном виде матричное уравнение примет вид:
$\begin {pmatrix}
	a'_x & a'_y & 1 \\
	b'_x & b'_y & 1 \\
	c'_x & c'_y & 1
\end {pmatrix} = \begin {pmatrix}
	a_x & a_y & 1 \\
	b_x & b_y & 1 \\
	c_x & c_y & 1
\end {pmatrix} \begin {pmatrix}
	1 & 0 & 0 \\
	0 & 1 & 0 \\
	k_x & k_y & 1
\end {pmatrix}$, или для трёхмерного пространства:

\begin {equation} \label {eq:translation}
	\begin {pmatrix}
		a'_x & a'_y & a'_z & 1 \\
		b'_x & b'_y & b'_z & 1 \\
		c'_x & c'_y & c'_z & 1
	\end {pmatrix} = \begin {pmatrix}
		a_x & a_y & a_z & 1 \\
		b_x & b_y & b_z & 1 \\
		c_x & c_y & c_z & 1
	\end {pmatrix} \begin {pmatrix}
		1 & 0 & 0 & 0 \\
		0 & 1 & 0 & 0 \\
		0 & 0 & 1 & 0 \\
		k_x & k_y & k_z & 1
	\end {pmatrix}
\end {equation}

\section {Поворот}
Перейдём к вращению фигуры в пространстве. В трёхмерном пространстве любое вращение можно задать с помощью угла поворота и оси вращения.
Поворот вокруг координатных осей может быть описан матрицей без однородных координат.
Для удобства рассмотрим поворот точки вокруг оси $z$ на угол $\gamma$ (угол $\alpha$ будет использоваться в непосредственных формулах далее, а пока мы рассматриваем "сферические формулы в вакууме", поэтому используем свободную букву):
$\begin {pmatrix}
	x' & y' & z'
\end {pmatrix} = \begin {pmatrix}
	x & y & z
\end {pmatrix} R_z$, где $R_z = \begin {pmatrix}
	cos(\gamma) & sin(\gamma) & 0 \\
	-sin(\gamma) & cos(\gamma) & 0 \\
	0 & 0 & 1
\end {pmatrix}$

Матрицы, определяющие поворот вокруг остальных осей можно получить из матрицы $R_z$ чисто формальным способом:

$R_x = \begin {pmatrix}
	1 & 0 & 0 \\
	0 & cos(\gamma) & sin(\gamma) \\
	0 & -sin(\gamma) & cos(\gamma) \\
\end {pmatrix}$ 
$R_y = \begin {pmatrix}
	cos(\gamma) & 0 & -sin(\gamma) \\
	0 & 1 & 0 \\
	sin(\gamma & 0 & cos(\gamma)
\end {pmatrix}$

Теперь можно найти матрицу $R$ для поворота вокруг прямой линии, проходящей через точку начала координат. Удобно полагать, что поворот осуществляется вокруг вектора $\overrightarrow{\nu}$, начало которого расположено в точке начала координат.

Если концевая точка вектора $\overrightarrow{\nu}$ задана в ортогональных координатах, то сначала вычислим его сферические координаты $\rho$, $\theta$ и $\varphi$. Зная формулы обратного исчисления:
$\nu_x = \rho sin(\varphi) cos(\theta), \nu_y = \rho sin(\varphi) sin(\theta), \nu_z = \rho cos(\varphi)$, получаем, что:
$\rho = |\overrightarrow{\nu}| = \sqrt{{\nu_x}^2 + {\nu_y}^2 + {\nu_z}^2}$, 
Если $\rho = 0$, то $\theta = \varphi = 0$. Иначе:
$\theta = \left\{
	\begin {aligned}
		\mathop{arctg(\nu_y/\nu_x),} \mathop{if} \nu_y > 0 \\
		\mathop{\pi + arctg(\nu_y/\nu_x),} \mathop{if} \nu_x < 0 \\
		\mathop{\pi/2,} \mathop{if} \mathop{\nu_x = 0,} \nu_y >= 0 \\
		\mathop{3\pi/2,} \mathop{if} \mathop{\nu_x = 0,} \nu_y < 0
	\end {aligned}
\right.$, $\varphi = arccos(\nu_z/\rho)$.

Сама суть осуществления поворота заключается в изменении системы координат так, чтобы вектор $\overrightarrow{\nu}$, являющийся осью вращения, совпадал с положительным направлением полуоси $z$. Так как мы будем вращать систему координат, а не саму фигуру, использоваться будут не матрицы $R_x$, $R_y$ и $R_z$, а обратные им ${R_x}^{-1}$, ${R_y}^{-1}$ и ${R_z}^{-1}$. Получение обратных матрица можно не расписывать, так как учитывая чётность/нечётность тригонометрических функций и правила построения обратных матриц их получение сведётся к простой замене знаков у синуса.

Начнём с поворота осей $x$ и $y$ вокруг оси $z$ на угол $\theta$:
$\begin {pmatrix} x' & y' & z' \end {pmatrix} = \begin {pmatrix} x & y & z \end {pmatrix} {R_z}^{-1}$, где ${R_z}^{-1} = \begin {pmatrix}
	cos(\theta) & -sin(\theta) & 0 \\
	sin(\theta) & cos(\theta) & 0 \\
	0 & 0 & 1
\end {pmatrix}$


Ось $x'$ имеет положительное направление вектора $(\nu_x. \nu_y, 0)$. Теперь следует повернуть оси $x'$ и $z'$ вокруг оси $y'$ на угол $\varphi$ до совпадения оси $z''$ с вектором $\overrightarrow{\nu}$:
$\begin {pmatrix} x'' & y'' & z'' \end {pmatrix} = \begin {pmatrix} x' & y' & z' \end {pmatrix} {R_y}^{-1}$, где ${R_y}^{-1} = \begin {pmatrix}
	cos(\varphi) & 0 & sin(\varphi) \\
	0 & 1 & 0 \\
	-sin(\varphi) & 0 & cos(\varphi)
\end {pmatrix}$


Теперь оставшийся поворот вокруг оси аппликат ($z''$) совпадает с поворотом вокруг вектора $\overrightarrow{\nu}$:
$\begin {pmatrix} x''' & y''' & z''' \end {pmatrix} = \begin {pmatrix} x'' & y'' & z'' \end {pmatrix} R_z = \begin {pmatrix} x'' & y'' & z'' \end {pmatrix} R_\nu$, где $R_\nu = \begin {pmatrix}
	cos(\alpha) & sin(\alpha) & 0 \\
	-sin(\alpha) & cos(\alpha) & 0 \\
	0 & 0 & 1
\end {pmatrix}$


Осталось выразить координаты $(x''', y''', z''')$ в изначальной системе координат, для чего используем матрицы $R_y$ и $R_z$:
$\begin {pmatrix} x^* & y^* & z^* \end {pmatrix} = \begin {pmatrix} x''' & y''' & z''' \end {pmatrix} R_y R_z$, где $R_y = \begin {pmatrix}
	cos(\varphi) & 0 & -sin(\varphi) \\
	0 & 1 & 0 \\
	sin(\varphi) & 0 & cos(\varphi)
\end {pmatrix}$, $R_z = \begin {pmatrix}
	cos(\theta) & sin(\theta) & 0 \\
	-sin(\theta) & cos(\theta) & 0 \\
	0 & 0 & 1
\end {pmatrix}$

Теперь выполнив обратную подстановку получим соотношение:
$\begin {pmatrix} x^* & y^* & z^* \end {pmatrix} = \begin {pmatrix} x & y & z \end {pmatrix} {R_z}^{-1} {R_y}^{-1} R_\nu R_y R_z = \begin {pmatrix} x & y & z\end {pmatrix} R$, где $R = {R_z}^{-1} {R_y}^{-1} R_\nu R_y R_z = \begin {pmatrix}
	r_{11} & r_{12} & r_{13} \\
	r_{21} & r_{22} & r_{23} \\
	r_{31} & r_{22} & r_{33}
\end {pmatrix}$ (такая запись будет удобна дальше).

Полученная матрица $R$ позволяет осуществить поворот относительно вектора, привязанного к точке начала системы координат. Остаётся избавиться от этого ограничения и найти матрицу поворота относительно любого вектора, начало которого расположено в произвольной точке $Q(q_x, q_y, q_z)$.
Данное действие относительно тривиально: на основании знаний предыдущего раздела осуществим перенос произвольной точки $Q$ в точку начала координат (для этого используем матрицу аналогичную $K^{-1}$, но вместо координат вектора $\overrightarrow{k}$ используя координаты точки $Q$), осуществим поворот относительно оси, которая теперь проходит через начало координат, но расширим матрицу $R$ для использования однородной системы координат, а затем осуществим обратное преобразование с помощью матрицы подобной $K$ (вместо координат вектора - координаты точки).

Таким образом матрица поворота вокруг произвольной оси примет вид: $R^+ = K^{-1} R^* K$, где $R^* = \begin {pmatrix}
	r_{11} & r_{12} & r_{13} & 0 \\
	r_{21} & r_{22} & r_{23} & 0 \\
	r_{31} & r_{22} & r_{33} & 0 \\
	0 & 0 & 0 & 1
\end {pmatrix}$, $K = \begin {pmatrix}
	1 & 0 & 0 & 0 \\
	0 & 1 & 0 & 0 \\
	0 & 0 & 1 & 0 \\
	q_x & q_y & q_z & 1
\end {pmatrix}$, $K^{-1} = \begin {pmatrix}
	1 & 0 & 0 & 0 \\
	0 & 1 & 0 & 0 \\
	0 & 0 & 1 & 0 \\
	-q_x & -q_y & -q_z & 1
\end {pmatrix}$

В конечном виде матричное уравнение поворота треугольника вокруг произвольной оси в пространстве примет вид:
\begin {equation} \label{eq:rotate}
	\begin {pmatrix}
		{a_x}^* & {a_y}^* & {a_z}^* & 1 \\
		{b_x}^* & {b_y}^* & {b_z}^* & 1 \\
		{c_x}^* & {c_y}^* & {c_z}^* & 1
	\end {pmatrix} = \begin {pmatrix}
		a_x & a_y & a_z & 1 \\
		b_x & b_y & b_z & 1 \\
		c_x & c_y & c_z & 1
	\end {pmatrix} R^+
\end {equation}

Полностью раскрывать все формулы и получать окончательный вид уравнения нет смысла: оно получается слишком объёмным и тяжёлым для восприятия.

\section {Подводя итоги}

Матричное уравнение для масштабирования треугольника в трёхмерном пространстве:
\begin {equation} \label {eq:scale2} \nonumber
	\begin {pmatrix}
		a'_x & a'_y & a'_z \\
		b'_x & b'_y & b'_z \\
		c'_x & c'_y & c'_z
	\end {pmatrix} = \begin {pmatrix}
		a_x & a_y & a_z \\
		b_x & b_y & b_z \\
		c_x & c_y & c_z
	\end {pmatrix} \begin {pmatrix}
		k_x & 0 & 0 \\
		0 & k_y & 0 \\
		0 & 0 & k_z
	\end {pmatrix}
\end {equation},
где $k_x$, $k_y$, $k_z$ - коэффициенты масштабирования по соответсвующей оси.

Матричное уравнения для переноса треугольника в трёхмерном пространстве:
\begin {equation} \label {eq:translation2} \nonumber
	\begin {pmatrix}
		a'_x & a'_y & a'_z & 1 \\
		b'_x & b'_y & b'_z & 1 \\
		c'_x & c'_y & c'_z & 1
	\end {pmatrix} = \begin {pmatrix}
		a_x & a_y & a_z & 1 \\
		b_x & b_y & b_z & 1 \\
		c_x & c_y & c_z & 1
	\end {pmatrix} \begin {pmatrix}
		1 & 0 & 0 & 0 \\
		0 & 1 & 0 & 0 \\
		0 & 0 & 1 & 0 \\
		k_x & k_y & k_z & 1
	\end {pmatrix}
\end {equation},
где $\overrightarrow{k}(k_x, k_y, k_z)$ - вектор переноса.

Матричное уравнение для поворота треугольника в трёхмерном пространстве:
\begin {equation} \label{eq:rotate2} \nonumber
	\begin {pmatrix}
		{a_x}^* & {a_y}^* & {a_z}^* & 1 \\
		{b_x}^* & {b_y}^* & {b_z}^* & 1 \\
		{c_x}^* & {c_y}^* & {c_z}^* & 1
	\end {pmatrix} = \begin {pmatrix}
		a_x & a_y & a_z & 1 \\
		b_x & b_y & b_z & 1 \\
		c_x & c_y & c_z & 1
	\end {pmatrix} K^{-1} R^* K
\end {equation},
где $K = \begin {pmatrix}
	1 & 0 & 0 & 0 \\
	0 & 1 & 0 & 0 \\
	0 & 0 & 1 & 0 \\
	q_x & q_y & q_z & 1
\end {pmatrix}$, $K^{-1} = \begin {pmatrix}
	1 & 0 & 0 & 0 \\
	0 & 1 & 0 & 0 \\
	0 & 0 & 1 & 0 \\
	-q_x & -q_y & -q_z & 1
\end {pmatrix}$, $R^* = \begin {pmatrix}
	r_{11} & r_{12} & r_{13} & 0 \\
	r_{21} & r_{22} & r_{23} & 0 \\
	r_{31} & r_{22} & r_{33} & 0 \\
	0 & 0 & 0 & 1
\end {pmatrix}$, которая в свою очередь расширяет матрицу поворота относительно начала координат $R = {R_z}^{-1} {R_y}^{-1} R_\nu R_y R_z$

\end{document}
